\documentclass{article}
\usepackage[utf8]{inputenc}
\usepackage[lmargin=2cm,tmargin=2cm,rmargin=2cm,bmargin=2cm]{geometry}
\usepackage[brazil]{babel}
\usepackage[T1]{fontenc}
\usepackage{amsfonts} 
\usepackage{amsmath, amsthm, amsfonts, amssymb, dsfont, mathtools, blindtext, polynom}
\usepackage{graphicx,color, xcolor}
\usepackage{verbatim}
\usepackage{pgfplots}
\pgfplotsset{width=10cm,compat=1.9}
\usepgfplotslibrary{external}
\usepackage{minted}
\usepackage{graphicx}
\usepackage{multirow}
\usepackage{tikz}
\usepackage{longtable}
\usetikzlibrary{calc,matrix}
\usetikzlibrary{shapes.geometric, arrows}
\tikzstyle{startstop} = [rectangle, rounded corners, minimum width=3cm, minimum height=1cm,text centered, draw=black, fill=red!30]
\usepackage{minted}

\newtheorem{defn}{Definição}
\newtheorem{exe}{Exemplo}
\newtheorem{ex}{Exercício}
\newtheorem{NOTA}{NOTA}
\newtheorem{obs}{Observação}
\newtheorem{fato}{Fato}
\newtheorem{teo}{Teorema}
\newtheorem{resp}{Resposta}
\DeclareMathOperator{\Hessian}{Hess}



\begin{document}
\title{% 
    Cálculo - Lista}
\author{Dyanna Cruz\\ Guilherme Camblor\\Renato Campos}
\date{\today}
\maketitle
\[\]

% ------------------------------------------------------------------
\begin{ex}
    Ache os pontos de máximo e/ou de mínimo da função $f(x,y)=x^4y^3$ sejeito à $x+y=1$.
\begin{resp}
\begin{align*}
    f(x,y)=x^4y^3 \to 4x^3y^3\text{ }e\text{ } x^43y^2\\
    g(x,y)=x+y-1\\
    \bigtriangledown f(x,y) = \\
    \\
    \begin{cases}
     4x^3y^3=\lambda\\
     x^43y^2=\lambda\\
     x+y-1=0
\end{cases}\\
    4x^3y^3=\lambda \to 4x^3y^3=x^43y^2 \to y=4x^33x^4 \to y=\frac{3}{4}x\\
    x+y-1=0\to x+\frac{3}{4}x=1 \to \frac{7x}{4}=1 \to 7x=4 \to x=\frac{4}{7}\\
    x+y-1=0 \to \frac{4}{7} + y = 1 \to y=1\cdot \frac{-4}{7} \to \frac{7-4}{7} \to y=\frac{3}{7}
\end{align*}

    Portanto o ponto $P=\left(\frac{4}{7},\frac{3}{7}\right)$ e a função no ponto $P=\frac{4^4\cdot 3^3}{7^7}$.

    Entretanto, dessa forma, não conseguimos observar se o ponto é de máximo ou mínimo. Vamos voltar e derivar uma segunda vez e analisar para onde o P tende.
\end{resp}
\end{ex}

% ------------------------------------------------------------------
\begin{ex}
    Ache os pontos críticos da função $f(x,y) = x^2+y^2+3xy-x+y$ e classifique-os (máximo, mínimo ou sela).
\begin{resp}
\begin{align*}
    f(x,y)=2x+3y-1 \text{ } e \text{ } 2y+3x+1\\
    \bigtriangledown f(x,y)=(2x+3y-1,2y+3x+1)=(0,0)
    \\
    \begin{cases}
    2x+3y=1\to 2x+3y-1=0 \\2y+3x=-1\to 2y+3x+1=0
    \end{cases}
\end{align*}
\begin{align*} %Eq 1 + Eq 2
    2x+3x+3y+2y=0\\
    5x+5y=0\\
    5x=-5y
    x=\frac{-5y}{5}\\
    x=-y
\end{align*}
\begin{align*} % subst. na 2
    2y+3x=-1\\
    2y+3\cdot(-y)=-1\\
    -y=-1\to y=1
\end{align*}
\begin{align*} % subst. na 2
    2x+3y=1\\
    2x+3\cdot 1=1\\
    2x+3=1\\
    2x=1-3\\
    2x=-2
    x=\frac{-2}{2}\\
    x=-1
\end{align*}

$P=(-1,1)$. Agora para poder classificar temos de encontrar o det da matriz Hessiana.

\begin{align*}
  (\Hessian f)_{ij} &\equiv \frac{\partial^{2} f}{\partial x_{i} \partial x_{j} } \to
  \Hessian\left( x_0,y_0 \right) = 
  \begin{bmatrix}
    \frac{\partial^{2} f}{\partial x^2}(x_0,y_0) & \frac{\partial^{2} f}{\partial x\partial y}(x_0,y_0) \\
    \frac{\partial^{2} f}{\partial x\partial y}(x_0,y_0) & \frac{\partial^{2} f}{\partial y^2}(x_0,y_0)
  \end{bmatrix}
\end{align*}
\begin{align*}
    f(x,y)=x^2+y^2+3xy-x+y\\
    f'(x)=2x+3y-1\\
    f''(y)=3\\
    \frac{\partial f}{\partial x \partial y}=3\\
    \Hessian =\begin{bmatrix}
    2 & 3 \\
    3 & 2
    \end{bmatrix}\\
    det=(2\cdot 2)-(3\cdot 3) = 4-9 = -5
\end{align*}

    Como o det é menor que zero, então podemos classificar que é um ponto de sela.

    Para achar o ponto crítico da função, temos que:
    \[f(-1,1)=-1^2+1^2+3\cdot(-1)\cdot(1)+1+1 \to 1+1-3+1+1 \to 4-3=1\]
    
    Logo o ponto crítico da função é (-1,1,1).
\end{resp}
\end{ex}

% ------------------------------------------------------------------
\begin{ex}
    Ache o(s) ponto(s) do plano $3x+y-z=1$ mais próximo de (1,1,1).
\begin{resp}
    
\end{resp}

\item Passo 1: Função objetiva:
    \begin{align*}
    T(x,y,z)&=f(x_0,y_0)+f(x_0,y_0)\cdot(x-x_0)+f(x_0,y_0)\cdot(y-y_0)\\
    f(x,y,z)&=x^2+1-2x+y^2-2y+1+z^2-2z+1\\
    f(x,y,z)&=x^2-2x+y^2-2y+z^2-2z+3\\
    g(x,y,z)&=3x+y-z-1 \to 3x+y-z=1
    \end{align*}
    
\item Passo 2: Método dos multiplicadores de Lagrange:
\begin{align*}
    &\begin{cases}
    f(x,y,z)=x^2-2x+y^2-2y+z^2-2z+3\\
    g(x,y,z)=3x+y-z-1\\ 
    \end{cases}\\
    \bigtriangledown f(x.y.z)&=\lambda \bigtriangledown g(x,y,z)\\
    \frac{\partial f}{\partial x}&=2x-2 \text{ }\text{ }\frac{\partial f}{\partial x}=2y-2\text{ }\text{ }\frac{\partial f}{\partial x}=2z-2\\
    \frac{\partial g}{\partial x}&=3\text{ }\text{ }\frac{\partial g}{\partial x}=1\text{ }\text{ }\frac{\partial g}{\partial x}=-1\\
    &\begin{cases}
    2x-2=3\lambda\\
    2y-2=\lambda\\
    2z-2=-\lambda
    \end{cases}
\end{align*}
\begin{align*}
    2x-2&=3\lambda\\
    2x&=3\lambda+2\\
    x&=\frac{3\lambda+2}{2}   
\end{align*}
\begin{align*}
    2y-2&=\lambda\\
    2y&=\lambda+2\\
    y&=\frac{\lambda+2}{2}   
\end{align*}
\begin{align*}
    2z-2&=-\lambda\\
    2z&=-\lambda+2\\
    z&=\frac{-\lambda+2}{2}   
\end{align*}
\[P(x,y,z)=P\left(\frac{3\lambda+2}{2},\frac{\lambda+2}{2},\frac{-\lambda+2}{2}\right)\]
\[P(x,y,z)=P\left(\frac{5}{11},\frac{9}{11},\frac{13}{11}\right)\]
\end{ex}

% ------------------------------------------------------------------
\begin{ex}
    (Regressão "parabólica")  
\item Dado um conjunto de treinamento $\left\{(0,1),(1,2),(2,9),(3,28), (-1,0),(-2,-7),(-3,-26)\right\}$. Ache a melhor parábola $\hat{y}=w_1x^2x+b$ de modo a minimizar a função de erro quadrático $E(w_1,w_2,b)$. Quanto seria y(5)?
\begin{resp}
\end{resp}
\begin{center}
\includegraphics[scale=0.7]{Q4.jpg}
\end{center}
\begin{center}
\includegraphics[scale=0.7]{Q4.2.jpg}
\end{center}
\begin{center}
\includegraphics[scale=0.7]{Q4.3.jpg}
\end{center}
\begin{center}
\includegraphics[scale=0.7]{Q4.4.jpg}
\end{center}
\end{ex}

% ------------------------------------------------------------------
\begin{ex}
    (Regressão "logística") Dado um conjunto de treinamento:
\item  $\left\{(0,1),(1,1),(2,1),(3,1), (4,1),(5,1),(6,1),(7,1),(8,1),(9,1),(10,1),(-1,1),(-2,0),(-3,0),(-4,0),(-5,0)\right\}$. 
\item Ache a melhor sigmóide $\hat{y}=\frac{1}{1+e^{-(wx+b)}}$ de modo a minimizar a função de erro quadrático $E(w,b)$. Quanto seria y(14) e $y(-7)$?
\begin{resp}
\end{resp}
\begin{center}
\includegraphics[scale=0.7]{Q5.jpg}
\end{center}
\begin{center}
\includegraphics[scale=0.7]{Q5.2.jpg}
\end{center}
\end{ex}

% ------------------------------------------------------------------
\begin{ex}
    Ache os pontos de mínimo da função $f(x,y)=x^2y^2$ sejeito à $x+y=1$ usando o algoritmo do gradiente descendente. Diga considere minimizar $H(x,y,\lambda)=	\bigtriangledown f(x,y)-\lambda	\bigtriangledown g(x,y)$.
\begin{resp}
\end{resp}
\begin{center}
\includegraphics[scale=0.7]{Q6.jpg}
\end{center}
\begin{center}
\includegraphics[scale=0.6]{Q6.2.jpg}
\end{center}
\begin{center}
\includegraphics[scale=0.6]{Q6.3.jpg}
\end{center}
\end{ex}

% ------------------------------------------------------------------
\begin{ex}
    Com base nos dados históricos do último ano, monte um portifólio de risco mínimo envolvendo as ações: ITUB4, BBSA4, BBDC4 e BCSA34.
\begin{resp}
\end{resp}
\begin{center}
\includegraphics[scale=0.7]{Q7.jpg}
\end{center}
\begin{center}
\includegraphics[scale=0.7]{Q7.2.jpg}
\end{center}
\end{ex}

% ------------------------------------------------------------------
%\begin{ex}
%    Resolva o exercício anterior com o método do gradiente descendente e usando a do exercício 6. Compare com o exercício anterior.
%\end{ex}


\end{document}